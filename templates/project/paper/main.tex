\documentclass[11pt, a4paper]{article}

% --- PACKAGES ---
\usepackage[utf8]{inputenc}
\usepackage[T1]{fontenc}
\usepackage{amsmath, amssymb, physics} % Standard Physics notation (bras, kets, etc.)
\usepackage{graphicx}                  % Images
\usepackage{geometry}                  % Margins
\usepackage{hyperref}                  % Hyperlinks
\usepackage{booktabs}                  % Professional tables
\usepackage{siunitx}                   % SI units handling
\usepackage{cleveref}                  % Smart referencing (e.g., "Figure 1")

% --- SETUP ---
\geometry{top=2.5cm, bottom=2.5cm, left=2.5cm, right=2.5cm}
\hypersetup{colorlinks=true, linkcolor=blue, citecolor=blue, urlcolor=blue}

% --- METADATA ---
\title{\textbf{Computational Analysis of System X}}
\author{Your Name \\ \textit{Department of Physics}}
\date{\today}

\begin{document}

\maketitle

\begin{abstract}
    \noindent We present a numerical study of the target system using C++/Fortran simulation. Data analysis indicates a strong correlation between input parameters and the phase space trajectory.
\end{abstract}

\section{Introduction}
The problem is defined by the Hamiltonian:
\begin{equation}
    H = \frac{p^2}{2m} + V(x)
\end{equation}
We solve this numerically using the methods described below.

\section{Methodology}
The simulation reads parameters from \texttt{input.txt} and evolves the system using a symplectic integrator.

\section{Results}
The resulting trajectory is visualized in \Cref{fig:trajectory}.

\begin{figure}[htbp]
    \centering
    % Robustly handle missing images for new projects
    \IfFileExists{../figures/plot.png}{
        \includegraphics[width=0.8\textwidth]{../figures/plot.png}
    }{
        \framebox{\parbox{0.8\textwidth}{\centering
            \vspace{2cm}
            \textbf{Figure Pending} \\
            \small Run \texttt{analysis/plot.py} to generate \texttt{figures/plot.png}
            \vspace{2cm}
        }}
    }
    \caption{Phase space trajectory generated by the analysis pipeline.}
    \label{fig:trajectory}
\end{figure}

\section{Conclusion}
The method proves robust for this regime.

\end{document}
